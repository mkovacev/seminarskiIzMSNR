% !TEX encoding = UTF-8 Unicode
\documentclass{beamer}

\usepackage{color}
\usepackage{url}
\usepackage[T2A]{fontenc}
\usepackage[utf8]{inputenc}
\usepackage{graphicx}
\usepackage[english,serbian]{babel}
\usepackage{chemfig}
%\usepackage[version=3]{mhchem}
%\usepackage{multicol}

\mode<presentation>
{
  \usetheme{Warsaw}       % or try default, Darmstadt, Warsaw, ...
  \usecolortheme{default} % or try albatross, beaver, crane, ...
  \usefonttheme{serif}    % or try default, structurebold, ...
  \setbeamertemplate{navigation symbols}{}
  \setbeamertemplate{caption}[numbered]
} 


\definecolor{mygreen}{rgb}{0,0.6,0}
\definecolor{mygray}{rgb}{0.5,0.5,0.5}
\definecolor{mymauve}{rgb}{0.58,0,0.82}

\usepackage{listings}
\lstset{ 
  backgroundcolor=\color{white},
  basicstyle=\scriptsize\ttfamily,
  breakatwhitespace=false,
  breaklines=true,
  captionpos=b,
  commentstyle=\color{mygreen},
  deletekeywords={...},            
  escapeinside={\%*}{*)},          
  extendedchars=true,
  firstnumber=1,              
  frame=single,	                
  keepspaces=true,
  keywordstyle=\color{blue},     
  language=Python,                
  morekeywords={*,...},
  numbers=left, 
  numbersep=4pt,                  
  numberstyle=\tiny\color{mygray}, 
  rulecolor=\color{black},
  showspaces=false,
  showstringspaces=false,
  showtabs=false,
  stepnumber=1, 
  stringstyle=\color{mymauve},
  tabsize=1,
  title=\lstname
}
\usepackage{pgfpages}
\pgfpagesuselayout{resize to}[%
  physical paper width=8in, physical paper height=6in]


% Here's where the presentation starts, with the info for the title slide
\title{Komunikacija preko mreže}
\author{K.Bakić M.Kovačević N.Milovanović N.Kovačević}
\date{\today}

\begin{document}

\begin{frame}
  \titlepage
\end{frame}

% These three lines create an automatically generated table of contents.

%---: SPISAK TEMA
\begin{frame}{Pregled}
 \tableofcontents
\end{frame}

\section{Uvod}
%prikaz na uskoj plavoj traci
	\begin{frame}{Uvod}
		\begin{itemize}
		\item dkskdjs
		\end{itemize}
	\end{frame}
	%section -----: Iznad, na crnom
\section{Izgubljeno poverenje}
	\subsection*{Vikipedija}
		\begin{frame}{Vikipedija}
			VIKIPEDIJA tekst
		\end{frame}

	\subsection*{Internet prevare}
		\begin{frame}{Internet prevare}
			\begin{itemize}
				\item Jednostavna sintaksa omogućava laku čitljivost koda
			\end{itemize}
		\end{frame}
	\subsection*{Društvene mreže}
		\begin{frame}{Društvene mreže}
			Društvene mreže tekst
		\end{frame}
	\subsection*{Podnaslov 4}
		\begin{frame}{Podnaslov 4}
		Društvene mreže tekst
		\end{frame}

\section{Zavisnost od interneta}
	\subsection*{Podnaslov1}
	\begin{frame}{Podnaslov1}
			PODNASLOV1 tekst
	\end{frame}

	\subsection*{Podnaslov2}
		\begin{frame}{Podnaslov2}
			\begin{itemize}
			\item Jednostavna sintaksa omogućava laku čitljivost koda
			\end{itemize}
		\end{frame}
	\subsection*{Podnaslov3}
	\begin{frame}{Podnaslov3}
		Podnaslov4 tekst
	\end{frame}
	
	\subsection*{Podnaslov4}
	\begin{frame}{Podnaslov4}
		Podnaslov4 tekst
	\end{frame}
	

\section{Literatura}

\begin{frame}{Literatura}

\bibliography{seminarski.bib}

\end{frame}

\end{document}
